\documentclass[10pt, aspectratio=169]{beamer}
\usepackage[utf8]{inputenc}
\usepackage{ctex}
\usepackage{listings}
\usepackage{xcolor}
\usepackage{booktabs}
\usepackage{graphicx}
\usepackage{amsmath}
\usepackage{algorithm}
\usepackage{algorithmic}
\usepackage{tikz}
\usetikzlibrary{shapes.geometric, arrows}

% 主题设置
\usetheme{Madrid}
\usecolortheme{whale}
\setbeamerfont{title}{size=\huge}
\setbeamerfont{subtitle}{size=\Large}
\setbeamerfont{frametitle}{size=\large}

% 代码块样式设置
\lstset{
    language=Python,
    basicstyle=\tiny\ttfamily,
    keywordstyle=\color{blue!80!black},
    stringstyle=\color{red!60!black},
    commentstyle=\color{green!50!black},
    breaklines=true,
    showstringspaces=false,
    frame=leftline,
    numbers=left,
    numberstyle=\tiny\color{gray},
    xleftmargin=1em,
    backgroundcolor=\color{gray!5}
}

\title[中科大校内文件搜索引擎]{基于 HBase 的分布式校内文件搜索引擎}
\subtitle{大数据系统基础实验报告}
\author{[小组名称]}
\institute[USTC]{中国科学技术大学}
\date{\today}

\begin{document}

% -----------------------------------------------------------------------------
% 封面与目录
% -----------------------------------------------------------------------------
\begin{frame}
    \titlepage
\end{frame}

\begin{frame}{演示概览}
    \tableofcontents
\end{frame}

% -----------------------------------------------------------------------------
% 1. 项目背景与架构
% -----------------------------------------------------------------------------
\section{项目背景与整体架构}

\begin{frame}{1.1 项目目标与挑战}
    \begin{columns}
        \column{0.5\textwidth}
        \textbf{核心目标}
        \begin{itemize}
            \item 构建覆盖中科大各院系网站的搜索引擎
            \item 实现非结构化文件(PDF/DOCX)的全文检索
            \item 搭建基于 HBase 的海量数据存储系统
        \end{itemize}
        
        \column{0.5\textwidth}
        \textbf{面临挑战}
        \begin{itemize}
            \item \alert{数据孤岛}:各院系网站结构差异巨大
            \item \alert{海量存储}:数万份文件对存储系统的扩展性要求高
            \item \alert{搜索质量}:如何从海量文本中精准召回相关文档
        \end{itemize}
    \end{columns}
\end{frame}

\begin{frame}{1.2 系统整体架构}
    \begin{itemize}
        \item \textbf{数据采集层 (Data Ingestion)}: Scrapy 分布式爬虫集群
        \item \textbf{数据存储层 (Storage)}: HBase 分布式数据库 (Thrift Server)
        \item \textbf{计算索引层 (Computation)}: 自研倒排索引构建器 + Jieba 分词
        \item \textbf{应用服务层 (Application)}: Flask Web Server + BM25 Ranking
    \end{itemize}
    \vspace{0.5cm}
    \begin{center}
        \textit{数据流向:Internet $\rightarrow$ Scrapy $\rightarrow$ HBase $\rightarrow$ Indexer $\rightarrow$ Search Engine}
    \end{center}
\end{frame}

% -----------------------------------------------------------------------------
% 2. 爬虫算法详解
% -----------------------------------------------------------------------------
\section{核心算法一:分布式爬虫设计}

\begin{frame}{2.1 爬虫策略设计}
    我们采用了 \textbf{混合遍历策略} 来平衡覆盖率与时效性:
    
    \vspace{0.3cm}
    
    \begin{block}{1. 广度优先搜索 (BFS)}
        \begin{itemize}
            \item \textbf{目的}:快速发现新的子域名和一级目录。
            \item \textbf{实现}:维护一个 URL 队列,优先提取页面中的 \texttt{<a>} 标签链接。
        \end{itemize}
    \end{block}
    
    \begin{block}{2. 定向深度优先 (Targeted DFS)}
        \begin{itemize}
            \item \textbf{目的}:针对包含 ``下载中心''、``附件''、``resource'' 等关键词的路径进行深度挖掘。
            \item \textbf{优势}:大幅提高了 PDF/DOC 等高价值文件的命中率。
        \end{itemize}
    \end{block}
\end{frame}

\begin{frame}[fragile]{2.2 增量去重与布隆过滤器}
    为避免重复抓取,我们在 Scrapy Middleware 中实现了去重逻辑:
    
    \begin{itemize}
        \item \textbf{URL 指纹}:对每个 URL 计算 MD5 哈希。
        \item \textbf{去重逻辑}:
        \begin{lstlisting}
import hashlib

def request_seen(self, request):
    fp = hashlib.md5(request.url.encode()).hexdigest()
    if fp in self.fingerprints:
        return True
    self.fingerprints.add(fp)
    return False
        \end{lstlisting}
        \item \textbf{优化方案}:对于海量 URL,可升级为 Bloom Filter(空间效率提升 10 倍以上)。
    \end{itemize}
\end{frame}

\begin{frame}{2.3 非结构化数据清洗 pipeline}
    针对 PDF/Word 文档的特殊处理流程:
    
    \begin{enumerate}
        \item \textbf{文件流下载}:检测 \texttt{Content-Type},如果是 application/pdf 则进入下载管道。
        \item \textbf{文本提取}:
            \begin{itemize}
                \item \texttt{pdfminer.six}:提取 PDF 文本层。
                \item \texttt{python-docx}:提取 Word 文档 XML 内容。
            \end{itemize}
        \item \textbf{噪声清洗}:
            \begin{itemize}
                \item 去除页眉页脚(正则表达式匹配重复出现的模式)。
                \item 去除乱码字符(过滤非 UTF-8 及控制字符)。
            \end{itemize}
    \end{enumerate}
\end{frame}

% -----------------------------------------------------------------------------
% 3. 数据管理算法详解
% -----------------------------------------------------------------------------
\section{核心算法二:数据管理与 HBase 设计}

\begin{frame}{3.1 为什么选择 HBase?}
    相比于 MySQL,HBase 在本场景下的优势:
    \begin{itemize}
        \item \textbf{列式存储}:网页数据稀疏,列族设计节省空间。
        \item \textbf{Schema-less}:适应不同网页结构的动态变化。
        \item \textbf{高吞吐写入}:LSM-Tree 架构支持爬虫的高并发写入。
    \end{itemize}
\end{frame}

\begin{frame}{3.2 RowKey 设计优化}
    \textbf{设计原则}:唯一性、散列性、长度控制。
    
    \vspace{0.3cm}
    
    \begin{alertblock}{Bad Case: 使用原始 URL 作为 RowKey}
        \begin{itemize}
            \item URL 长度不一,导致存储开销大。
            \item 相似 URL 前缀相同(如 \texttt{ustc.edu.cn/...}),导致 Region Server 热点问题,无法利用集群并行能力。
        \end{itemize}
    \end{alertblock}
    
    \begin{exampleblock}{Our Solution: Hash + Timestamp}
        \texttt{RowKey = MD5(URL)[0:8] + "\_" + Timestamp}
        \begin{itemize}
            \item \textbf{MD5前缀}:确保数据随机分布到不同 Region,实现负载均衡。
            \item \textbf{Timestamp}:支持同一 URL 的多版本存储(历史快照)。
        \end{itemize}
    \end{exampleblock}
\end{frame}

\begin{frame}[fragile]{3.3 HBase 表结构定义}
    我们设计了单表多列族的结构:\texttt{ustc\_documents}
    
    \begin{table}
        \centering
        \begin{tabular}{l|l|l}
            \toprule
            \textbf{Row Key} & \textbf{Column Family: info} & \textbf{Column Family: content} \\
            \midrule
            \texttt{a1b2c3d4\_167...} & \texttt{info:url} = https://... & \texttt{content:html} = <html>... \\
             & \texttt{info:title} = 教务通知 & \texttt{content:text} = 清洗后文本... \\
             & \texttt{info:date} = 2024-01-15 & \\
            \bottomrule
        \end{tabular}
    \end{table}
    
    \begin{itemize}
        \item \textbf{info}: 存储元数据,常驻内存,访问频率高。
        \item \textbf{content}: 存储大文本,按需读取,节省 I/O。
    \end{itemize}
\end{frame}

% -----------------------------------------------------------------------------
% 4. 推荐与搜索算法详解
% -----------------------------------------------------------------------------
\section{核心算法三:搜索排序与相关度计算}

\begin{frame}{4.1 倒排索引 (Inverted Index) 构建}
    为了实现 $O(1)$ 时间复杂度的查询,我们构建了倒排索引:
    
    \vspace{0.3cm}
    \textbf{Term Dictionary (词典)} $\rightarrow$ \textbf{Posting List (倒排表)}
    
    \begin{itemize}
        \item \textbf{Key}: \texttt{计算机} (分词后的 Token)
        \item \textbf{Value}: \texttt{[DocID\_1, DocID\_5, DocID\_9, ...]}
    \end{itemize}
    
    \vspace{0.3cm}
    \textbf{流程}:
    \begin{enumerate}
        \item \textbf{分词}:使用 Jieba 引擎 + 自定义词典(``中科大'',``少年班'')。
        \item \textbf{停用词过滤}:去除 ``的''、``了''、``是'' 等无意义高频词。
        \item \textbf{索引合并}:将所有文档的 Token 映射关系持久化到 HBase 的 \texttt{index} 表中。
    \end{enumerate}
\end{frame}

\begin{frame}{4.2 排序算法演进:从 TF-IDF 到 BM25}
    \begin{columns}
        \column{0.45\textwidth}
        \textbf{TF-IDF 的局限}
        \begin{itemize}
            \item 词频线性增长,未考虑饱和度。
            \item 长文档天然容易包含更多关键词,导致分数偏高(长文档偏差)。
        \end{itemize}
        
        \column{0.55\textwidth}
        \textbf{BM25 的改进 (Best Matching)}
        \begin{itemize}
            \item \textbf{词频饱和} ($k_1$):当词频达到一定程度,分数增长变缓。
            \item \textbf{长度归一化} ($b$):惩罚过长的文档。
        \end{itemize}
    \end{columns}
    
    \vspace{0.5cm}
    \begin{equation*}
        Score(D, Q) = \sum_{i=1}^{n} IDF(q_i) \cdot \frac{f(q_i, D) \cdot (k_1 + 1)}{f(q_i, D) + k_1 \cdot (1 - b + b \cdot \frac{|D|}{avgdl})}
    \end{equation*}
    \centering \tiny{本项目参数配置:$k_1=1.5, b=0.75$}
\end{frame}

\begin{frame}[fragile]{4.3 最终 Ranking 策略:多因子融合}
    单一的 BM25 往往不够,我们引入了 \textbf{标题加权 (Title Boosting)} 策略:
    
    \begin{block}{评分公式}
        $$ FinalScore = BM25(Content) \times TitleWeight $$
    \end{block}
    
    \textbf{代码实现逻辑}:
    \begin{lstlisting}
# 1. 计算内容匹配度
score = ranker.calculate_bm25(doc_tokens, query_tokens)

# 2. 计算标题匹配度
title_tokens = tokenize(doc.title)
match_ratio = len(set(query) & set(title)) / len(query)

# 3. 动态加权
if match_ratio > 0.8:
    score *= 2.0  # 强匹配,权重翻倍
elif match_ratio > 0.5:
    score *= 1.5  # 弱匹配,适度提升
    \end{lstlisting}
\end{frame}

% -----------------------------------------------------------------------------
% 5. 总结
% -----------------------------------------------------------------------------
\section{总结与展望}

\begin{frame}{5.1 实验成果总结}
    \begin{itemize}
        \item \textbf{工程能力}:完整实现了从数据采集、清洗、存储到检索的全链路大数据系统。
        \item \textbf{算法深度}:没有依赖现成的 ES,手写实现了工业级搜索算法 BM25,深入理解了相关性计算原理。
        \item \textbf{系统架构}:通过 HBase 解决了海量非结构化数据的存储难题。
    \end{itemize}
\end{frame}

\begin{frame}{5.2 待改进方向}
    \begin{enumerate}
        \item \textbf{分布式扩展}:目前的爬虫和 Web 服务仍是单节点,可利用 ZooKeeper 进行分布式协调。
        \item \textbf{语义搜索}:引入 Word2Vec 或 BERT 向量模型,解决关键词不匹配但语义相关的问题(如搜 ``计算机'' 也能出 ``CS'')。
        \item \textbf{实时索引}:目前索引构建是离线的,可引入 Kafka 实现数据写入即搜索(NRT)。
    \end{enumerate}
\end{frame}

\begin{frame}
    \centering
    \Huge
    \textbf{感谢聆听!}
    
    \vspace{1cm}
    \Large
    请老师批评指正
\end{frame}

\end{document}

