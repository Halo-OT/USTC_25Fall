\documentclass[10pt]{beamer}
\usepackage[utf8]{inputenc}
\usepackage{ctex}
\usepackage{listings}
\usepackage{xcolor}
\usepackage{booktabs}
\usepackage{graphicx}
\usepackage{amsmath}

\usetheme{Madrid}
\usecolortheme{whale}

% 代码块样式设置
\lstset{
    language=Python,
    basicstyle=\tiny\ttfamily,
    keywordstyle=\color{blue},
    stringstyle=\color{red},
    commentstyle=\color{green!60!black},
    breaklines=true,
    showstringspaces=false,
    frame=single,
    numbers=left,
    numberstyle=\tiny\color{gray}
}

\title[中科大校内文件搜索引擎]{基于 HBase 的分布式校内文件搜索引擎}
\subtitle{大数据系统基础实验报告}
\author{[小组名称]}
\institute[USTC]{中国科学技术大学}
\date{\today}

\begin{document}

% 封面页
\begin{frame}
    \titlepage
\end{frame}

% 目录页
\begin{frame}{目录}
    \tableofcontents
\end{frame}

% 第一部分:小组成员与分工
\section{小组成员与分工}
\begin{frame}{小组成员与分工}
    \begin{table}
        \centering
        \small
        \begin{tabular}{llp{7cm}}
            \toprule
            \textbf{姓名} & \textbf{分工角色} & \textbf{具体职责} \\
            \midrule
            [姓名A] & 组长 & 项目架构设计、HBase集群搭建、分布式爬虫开发 \\
            [姓名B] & 算法架构 & 搜索引擎核心实现(BM25/倒排索引)、存储层封装 \\
            [姓名C] & 全栈开发 & Flask Web后端、前端界面、测试与文档撰写 \\
            \bottomrule
        \end{tabular}
    \end{table}
\end{frame}

% 第二部分:技术路线
\section{技术路线}

\begin{frame}{技术路线概览}
    \begin{itemize}
        \item \textbf{数据采集}:Scrapy 爬虫
        \begin{itemize}
            \item 广度优先搜索 (BFS) 遍历校内站点
            \item 深度优先抓取“下载中心”文件
            \item 解析 PDF/DOCX 非结构化数据
        \end{itemize}
        \vspace{0.5cm}
        \item \textbf{数据存储}:HBase 分布式数据库
        \begin{itemize}
            \item \textbf{Row Key}:\texttt{MD5(URL)[:8] + Timestamp}
            \item \textbf{列族}:\texttt{info} (元数据), \texttt{index} (索引)
            \item 实现了 Local File Fallback 机制
        \end{itemize}
    \end{itemize}
\end{frame}

\begin{frame}{搜索引擎核心算法}
    \begin{block}{BM25 相关度评分}
        相比传统 TF-IDF,引入词频饱和与长度归一化:
        \begin{equation*}
            Score(D, Q) = \sum_{i=1}^{n} IDF(q_i) \cdot \frac{f(q_i, D) \cdot (k_1 + 1)}{f(q_i, D) + k_1 \cdot (1 - b + b \cdot \frac{|D|}{avgdl})}
        \end{equation*}
        \footnotesize{参数设置:$k_1=1.5$ (饱和度), $b=0.75$ (长度惩罚)}
    \end{block}
    
    \vspace{0.3cm}
    
    \begin{block}{标题加权策略 (Title Weighting)}
        \begin{itemize}
            \item 查询词在标题中出现比例 $>80\%$ $\rightarrow$ 权重 $\times 2.0$
            \item 查询词在标题中出现比例 $>50\%$ $\rightarrow$ 权重 $\times 1.5$
        \end{itemize}
    \end{block}
\end{frame}

\begin{frame}{系统架构}
    \textbf{用户界面}:Flask + Bootstrap (响应式设计)
    
    \vspace{0.5cm}
    
    \begin{itemize}
        \item \textbf{轻量级后端}:解析请求 -> 查询 HBase -> 排序结果
        \item \textbf{多端适配}:完美支持 PC 与移动端访问
    \end{itemize}
\end{frame}

% 第三部分:功能与效果
\section{功能与效果展示}
\begin{frame}{核心功能}
    \begin{enumerate}
        \item \textbf{深度爬取能力}
        \begin{itemize}
            \item 穿透多级目录,直达“附件下载”
            \item 支持 pdf, doc, xls 等多种格式解析
        \end{itemize}
        \vspace{0.3cm}
        \item \textbf{精准搜索}
        \begin{itemize}
            \item 支持按“来源网站”筛选 (e.g., 教务处, 财务处)
            \item 毫秒级响应 (基于倒排索引)
        \end{itemize}
        \vspace{0.3cm}
        \item \textbf{智能排序}
        \begin{itemize}
            \item 基于 BM25 + 标题加权,结果更符合直觉
        \end{itemize}
    \end{enumerate}
\end{frame}

% 第四部分:核心代码
\section{核心代码实现}
\begin{frame}[fragile]{HBase 数据写入接口}
    \begin{lstlisting}
def _save_to_hbase(self, doc: Document, row_key: str) -> str:
    try:
        table = self.connection.table(self.table_name)
        data = doc.to_dict()
        # 准备HBase数据,存储在 info 列族下
        hbase_data = {}
        for key, value in data.items():
            if value:
                hbase_data[f'info:{key}'] = str(value).encode('utf-8')
        
        table.put(row_key.encode(), hbase_data)
        return row_key
    except Exception as e:
        print(f"Error saving to HBase: {e}")
        raise
    \end{lstlisting}
\end{frame}

\begin{frame}[fragile]{BM25 算法实现}
    \begin{lstlisting}
def calculate_bm25(self, doc_tokens: List[str], query_tokens: List[str]) -> float:
    score = 0.0
    doc_length = len(doc_tokens)
    doc_token_freq = Counter(doc_tokens)
    
    for query_token in query_tokens:
        if query_token in doc_token_freq:
            tf = doc_token_freq[query_token]
            # IDF * (TF部分 * 饱和度控制) / (长度归一化)
            numerator = self.idf_cache.get(query_token, 0) * tf * (self.k1 + 1)
            denominator = tf + self.k1 * (1 - self.b + self.b * (doc_length / max(self.avg_doc_length, 1)))
            score += numerator / max(denominator, 1)
    return score
    \end{lstlisting}
\end{frame}

% 第五部分:总结与心得
\section{总结与心得}
\begin{frame}{踩坑与收获}
    \begin{columns}[t]
        \column{0.33\textwidth}
        \textbf{[姓名A] (架构)}
        \begin{itemize}
            \item \textbf{问题}:Thrift 接口未启动导致连接失败
            \item \textbf{收获}:深入理解了 HBase 的部署与服务治理
        \end{itemize}
        
        \column{0.33\textwidth}
        \textbf{[姓名B] (算法)}
        \begin{itemize}
            \item \textbf{问题}:结果冗余
            \item \textbf{解决}:引入 SimHash 去重
            \item \textbf{收获}:彻底掌握了从 TF-IDF 到 BM25 的演进
        \end{itemize}
        
        \column{0.33\textwidth}
        \textbf{[姓名C] (全栈)}
        \begin{itemize}
            \item \textbf{问题}:PDF 乱码与清洗
            \item \textbf{收获}:掌握了非结构化数据的处理流程与 Linux 运维
        \end{itemize}
    \end{columns}
\end{frame}

\begin{frame}
    \centering
    \Huge
    \textbf{感谢聆听!}
    
    \vspace{1cm}
    \large
    Q \& A
\end{frame}

\end{document}

